%\input{preamble}
%%\usepackage{wallpaper}                                          % 使用浮水印
%%\CenterWallPaper{0.6}{images/ntpu.eps}                           % 浮水印圖檔
%\begin{document}
%\fontsize{12}{22pt}\selectfont
%\cleardoublepage
%\thispagestyle{empty}
%\setlength{\parindent}{2em}


\chapter{研究結果}

	本章節第一節會先簡單以文字雲呈現評論資料的文字雲,再以表格說明以TF-IDF和Conditional Entropy兩種方法所結合得到的關鍵字。第二節則是使用冰島和希臘這兩個國家的旅遊評論資料,對於三種詞向量模型配上三種機器學習分類模型,去比較模型預測結果和表現,第三節是透過前二節的方法結果,建立冰島和希臘兩個國家的推薦模型,依據找出的負面評論旅客,對其做出特定國家旅館推薦。
	

\section{特徵字}

	經過資料前處理,先將兩個國家的旅館評論,以文字雲呈現如圖\ref{Fig1},再將真實負面旅客的評論做成文字雲呈現。圖\ref{Fig2},可以看到只取負面評論的文字雲,與不分情感的文字雲,以兩種不同的國家,其畫出來的結果都很類似,也說明了常出現在評論的就是那幾個字,只考慮頻率高的字,很難確認是否為負面評論中的關鍵字,例如:「breakfast」、「staff」、「small」,都是在正反情緒都有出現的高頻字。而一些正面的單詞也會出現在在圖\ref{Fig2}裡面,像是「good」、「nice」,這裡給出兩種原因,第一可能是文字雲只考慮單字,像是「good」在評論裡面會寫成「not good」,第二種是資料在定義標籤時,評論內只要出現負面詞句以及給低分,就給予真實負面評論旅客的標籤,但在批評旅館的同時,也在認為旅館表現好的地方給予稱讚。
	
\newpage

	\begin{figure}[h!]
	\centering
	\subfloat[冰島]{\includegraphics[scale=0.3]{output/冰島文字雲.png}}
	\subfloat[希臘]{\includegraphics[scale=0.3]{output/希臘文字雲.png}}
	\caption{旅館評論文字雲}\label{Fig1}
	\end{figure}
	
	\begin{figure}[h!]
	\centering
	\subfloat[冰島]{\includegraphics[scale=0.3]{output/冰島文字雲(不好).png}}
	\subfloat[希臘]{\includegraphics[scale=0.3]{output/希臘文字雲(不好).png}}
	\caption{旅客負面評論文字雲}\label{Fig2}
	\end{figure}

\noindent 以圖\ref{Fig3}為例,就是第二種情況,這則旅客評論同時有正面評論也有負面評論,且給旅館分數為4.6分,所以有褒有貶的情況下,正面字詞「good」將被納入文字雲的計算。因為旅客在留言時屬於主觀的,對於喜歡的服務和設施是給予稱讚,也會抱怨不喜歡的人事物,假設旅館沒有很明顯很大的缺陷下,不太可能所有旅客評論全部都留下負面的字眼,總而言之負面評論也會出現正面回饋。
	
	\begin{figure}[H]
	\centering
	\includegraphics[scale=1.1]{論文內放的截圖/褒貶評論.png}
	\caption{褒貶評論}
	\label{Fig3}
	\end{figure}

	因為文字雲這個方式只能查看一個單詞在全部評論中出現的次數,我們不能確定最多出現次數的「breakfast」在旅客評論中,是稱讚早餐的好還是抱怨早餐,所以改用TF-IDF和Conditional Entropy來找尋特徵字,使用TF-IDF的好處,能夠考慮到這個評論中單詞的逆文檔頻率,還有一個優點可以考慮到兩個詞以上,像是「good breakfast」和「bad breakfast」,這樣就能分辨出「breakfast」是否為負面評論關鍵字,以及Conditional Entropy的方法,去查看這個關鍵字,在分正面和負面兩類時候的不確定程度。 
	
	由冰島的評論資料找出的30個負面關鍵字,將結果列在表\ref{tab1},從左邊的一個單詞可以看到前幾名的關鍵字,都是較為負面的詞,但因為都是形容詞,無法具體的說明旅館遇到的問題,則考慮二個單詞以上的結果,能發現冰島旅館被抱怨的具體問題,像是負面關鍵字「dead fly」,旅館就能透過改善衛生環境,降低負面評論的可能性,去提升飯店的品質。而希臘30個負面關鍵字可以看到表\ref{tab2},關鍵字「rude staff」說明了旅客對於旅館員工的行為抱持著厭惡的態度,旅館需要去加強員工的訓練,以防旅客遇到不好的服務體驗。
	
	\begin{table}[H] 
	\centering
	\caption{冰島旅館30個負面關鍵字}
	\begin{tabular}{|cc||cc|}
	\toprule
	一個單詞 & Conditional Entropy & 二個單詞以上 & Conditional Entropy \\
	\midrule
	terrible & 0.31755 & dead fly & 0.11992\\
	\midrule
	dirty & 0.32834 & bad place & 0.13673 \\
	\midrule
	unpleasant & 0.43819 & bad value money & 0.15765 \\
	\midrule
	poor & 0.45327 & bad value  & 0.15972 \\
	\midrule
	broken & 0.4548 & bad experience & 0.1847 \\
	\midrule
	impossible & 0.49224 & bad hotel & 0.18562 \\
	\midrule
	bad & 0.51742 & dirty bathroom & 0.19334 \\
	\midrule
	sheet & 0.52137 & unfriendly staff & 0.19334 \\
	\midrule
	uncomfortable & 0.53064 & dirty room & 0.21009 \\
	\midrule
	tiny & 0.61871 & room dirty & 0.22408 \\
	\midrule
	picture & 0.62441 & bathroom dirty & 0.24231 \\
	\midrule
	photo & 0.63214 & smell bad & 0.27793 \\
	\midrule
	smell & 0.64745 & bad room & 0.30879 \\
	\midrule
	customer & 0.67065 & impossible sleep & 0.32665 \\
	\midrule
	wall & 0.6823 & light bulb & 0.34353 \\
	\midrule
	furniture & 0.683 & poor quality & 0.36247 \\
	\midrule
	tell & 0.68941 & bad smell & 0.36693 \\
	\midrule
	pay & 0.71823 & noisy hear & 0.37072 \\
	\midrule
	mattress & 0.71902 & really bad & 0.38042 \\
	\midrule
	hair & 0.7228 & uncomfortable bed & 0.38149 \\
	\midrule
	great & 0.72562 & tiny room & 0.40091 \\
	\midrule
	say & 0.74063 & worth price & 0.42407 \\
	\midrule
	possible & 0.74304 & extremely noisy & 0.43073 \\
	\midrule
	change & 0.74436 & non existent & 0.43361 \\
	\midrule
	loud & 0.77006 & euro night & 0.43468 \\
	\midrule
	ask & 0.77198 & breakfast poor & 0.437 \\
	\midrule
	noisy & 0.77301 & poor breakfast & 0.44083 \\
	\midrule
	despite & 0.774 & bed uncomfortable & 0.46145 \\
	\midrule
	elevator & 0.77902 & previous guest & 0.49311 \\
	\midrule
	old & 0.78096 & extremely small & 0.49806 \\
	\bottomrule
	\end{tabular}
	\label{tab1}   
	\end{table}
	
	\begin{table}[H] 
	\centering
	\caption{希臘旅館30個負面關鍵字}
	\begin{tabular}{|cc||cc|}
	\toprule
	一個單詞 & Conditional Entropy & 二個單詞以上 & Conditional Entropy \\
	\midrule
	rude & 0.2853 & rude staff &  0.07794 \\
	\midrule
	horrible & 0.29051 & dirty room &  0.13414 \\
	\midrule
	awful & 0.31702 & poor condition & 0.16379 \\
	\midrule
	dirty & 0.36849 & dirty bathroom & 0.17333 \\
	\midrule
	terrible & 0.42915 & staff rude & 0.18418 \\
	\midrule
	broken & 0.47214 & room dirty & 0.19665 \\
	\midrule
	bad & 0.50844 & bad experience & 0.22343 \\
	\midrule
	unpleasant & 0.52643 & bad hotel & 0.261 \\
	\midrule
	poor & 0.55714 & bad smell & 0.26472 \\
	\midrule
	dangerous & 0.57023 & hotel bad & 0.32387 \\
	\midrule
	smell & 0.58087 & bad breakfast & 0.34201 \\
	\midrule
	uncomfortable & 0.5902 & bad location & 0.34707 \\
	\midrule
	photo & 0.59026 & room bad & 0.36306 \\
	\midrule
	drug & 0.61226 & furniture old & 0.38852 \\
	\midrule
	tell & 0.63318 & extremely noisy & 0.39885 \\
	\midrule
	great & 0.64665 & poor quality & 0.40679 \\
	\midrule
	impossible & 0.67377 & low quality & 0.4127 \\
	\midrule
	picture & 0.67726 & really bad & 0.41583 \\
	\midrule
	tiny & 0.6998 & like picture & 0.41734 \\
	\midrule
	sheet & 0.70318 & good thing & 0.42334 \\
	\midrule
	pay & 0.70902 & uncomfortable bed & 0.42739 \\
	\midrule
	conditioner & 0.7195 & tiny room & 0.42779 \\
	\midrule
	charge & 0.73257 & bad room & 0.43593 \\
	\midrule
	helpful & 0.73558 & bed uncomfortable & 0.44907 \\
	\midrule
	card & 0.73633 & breakfast bad & 0.48655 \\
	\midrule
	book & 0.74625 & breakfast poor & 0.49029 \\
	\midrule
	furniture & 0.75137 & room poor & 0.50474 \\
	\midrule
	sleep &  0.75457 & location bad & 0.51932 \\
	\midrule
	wall & 0.7615 & impossible sleep & 0.52035 \\
	\midrule
	toilet & 0.76447 & smell room & 0.52184 \\
	\bottomrule
	\end{tabular}
	\label{tab2}   
	\end{table}
	   
	
\section{模型結果}

	不同的詞嵌入方法配合三種機器學習的模型,可能會影響最後的預測結果。在預測之前,這裡先將資料分為80\%的訓練集與20\%的測試集,而資料則採用兩個國家的旅客評論資料去討論。因為是不平衡資料,第2.1小節考慮到類別權重,則對模型中的少數類別調高類別權重,這裡因為少數類別和多數類別比是1:32,則權重選擇用10到100,每次增加10,觀察是否會影響預測的結果,這裡分別以Random Forest和SVM去比較模型結果,模型使用Recall去衡量好壞。第2.2小節則加入GBDT模型,加上2.1小節的結果,模型指標則加入F1 score和AUC score,去比較和選擇最佳的模型組合。
	
\subsection{類別權重}

	這筆資料使用冰島旅客評論,從表\ref{tab3}中隨機森林的結果可以看到,當配上CBOW模型時,比其他三種方法較好一點,而且隨著權重的提高,可以看到Recall有變高的趨勢,當類別權重為100時,可以得到最高的Recall為0.06564,而最低的是配上Skip gram模型且類別權重只有10的情況下,Recall只有0.00875。
	
	\begin{table}[H]
	\centering
	\caption{隨機森林不同權重下的Recall(冰島)}
	\begin{tabular}{|c|cccc|}
	\toprule
	 Weight & BOW+RF & TFIDF+RF & CBOW+RF & SKIP+RF \\
	\midrule
	10 & 0.02697 & 0.01659 & 0.02844 & 0.00875 \\
	\midrule
	20 & 0.02489 & 0.02904 & 0.04157 & 0.01969 \\
	\midrule
	30 & 0.02904 & 0.02697 & 0.04814 & 0.03063 \\
	\midrule
	40 & 0.03112 & 0.02697 & 0.05689 & 0.03063 \\
	\midrule
	50 & 0.02697 & 0.03319 & 0.05470 & 0.03938 \\
	\midrule
	60 & 0.03319 & 0.03319 & 0.05689 & 0.03282 \\
	\midrule
	70 & 0.03319 & 0.03734 & 0.06126 & 0.03719 \\
	\midrule
	80 & 0.03112 & 0.03319 & 0.05908 & 0.03719 \\
	\midrule
	90 & 0.03941 & 0.04356 & 0.06345 & 0.03719 \\
	\midrule
	100 & 0.03941 & 0.03734 & 0.06564 & 0.04157 \\
	\bottomrule
	\end{tabular}
	\label{tab3}
	\end{table}
	
從表\ref{tab4}中SVM的結果可以看到,SVM的表現在這筆資料的預測結果,明顯優於隨機森林很多,而當SVM配上Skip gram模型時可以得到最佳表現,但CBOW模型的表現也不錯,Recall的表現也是將近6成多,再來是詞袋模型,表現最差的是TF-IDF模型,權重的部分跟隨機森林一樣,Recall也有變高的趨勢,當類別權重超過70的Skip gram模型,會得到Recall為0.73960。
	
	\begin{table}[H]
	\centering
	\caption{SVM不同權重下的Recall(冰島)}
	\begin{tabular}{|c|cccc|}
	\toprule
	 Weight & BOW+SVM & TFIDF+SVM & CBOW+SVM & SKIP+SVM \\
	\midrule
	10 & 0.21784 & 0.18879 & 0.64332 & 0.68708 \\
	\midrule
	20 & 0.26348 & 0.18879 & 0.67833 & 0.73304 \\
	\midrule
	30 & 0.27385 & 0.18879 & 0.67614 & 0.73741 \\
	\midrule
	40 & 0.27800 & 0.18879 & 0.67833 & 0.73741 \\
	\midrule
	50 & 0.28838 & 0.18879 & 0.67833 & 0.73304 \\
	\midrule
	60 & 0.28630 & 0.18879 & 0.67833 & 0.73522 \\
	\midrule
	70 & 0.29045 & 0.18879 & 0.67833 & 0.73960 \\
	\midrule
	80 & 0.28838 & 0.19087 & 0.67833 & 0.73960 \\
	\midrule
	90 & 0.28215 & 0.19087 & 0.67833 & 0.73960 \\
	\midrule
	00 & 0.28630 & 0.19087 & 0.67833 & 0.73960 \\
	\bottomrule
	\end{tabular}
	\label{tab4}
	\end{table}
	
最後將表\ref{tab3}和表\ref{tab4}的結果,繪製在圖\ref{Fig4}上,在使用冰島的資料下,可以清楚的發現SVM的模型不管配上哪種詞向量模型,Recall的表現都比隨機森林這個模型好,對於不平衡的文本情緒分類,SVM處理優於隨機森林,而調整類別權重會影響到Recall,。
		
	\begin{figure}[H]
	\centering
	\includegraphics[scale=0.45]{output/冰島模型比較.png}
	\caption{冰島旅館模型(權重 vs. Recall)}
	\label{Fig4}
	\end{figure}
	
	使用另一筆資料希臘旅客評論,從表\ref{tab5}中隨機森林的結果可以看到,Word2Vec的兩種模型在這裡都表現較差,而詞袋模型普遍在多個不同權重下表現較好,而當隨著權重提高到100時,可以看到TF-IDF模型的Recall達到最高的0.07407。
	
	\begin{table}[H]
	\centering
	\caption{隨機森林不同權重下的Recall(希臘)}
	\begin{tabular}{|c|cccc|}
	\toprule
	 Weight & BOW+RF & TFIDF+RF & CBOW+RF & SKIP+RF \\
	\midrule
	10 & 0.04320 & 0.01851 & 0.01293 & 0.01293 \\
	\midrule
	20 & 0.04732 & 0.03292 & 0.02586 & 0.01508 \\
	\midrule
	30 & 0.05349 & 0.03497 & 0.02370 & 0.01293 \\
	\midrule
	40 & 0.05967 & 0.03703 & 0.03448 & 0.01939 \\
	\midrule
	50 & 0.05349 & 0.04115 & 0.03232 & 0.01508 \\
	\midrule
	60 & 0.05144 & 0.04526 & 0.04094 & 0.01293 \\
	\midrule
	70 & 0.05349 & 0.05555 & 0.02801 & 0.02155 \\
	\midrule
	80 & 0.06172 & 0.05349 & 0.03448 & 0.02155 \\
	\midrule
	90 & 0.06172 & 0.05349 & 0.03879 & 0.02370 \\
	\midrule
	100 & 0.06584 & 0.07407 & 0.03663 & 0.02155 \\
	\bottomrule
	\end{tabular}
	\label{tab5}
	\end{table}
	
\newpage
	
從表\ref{tab6}中SVM的結果可以看到,SVM的表現在這筆資料,也是明顯優於隨機森林很多,而當SVM配上Word2Vec的兩種模型表現較好,再來是詞袋模型,表現最差的是TF-IDF模型,權重的部分也跟隨機森林一樣,Recall也有變高的趨勢,當類別權重超過60的Skip gram模型,會得到Recall為0.69827。

	\begin{table}[H]
	\centering
	\caption{SVM不同權重下的Recall(希臘)}
	\begin{tabular}{|c|cccc|}
	\toprule
	 Weight & BOW+SVM & TFIDF+SVM & CBOW+SVM & SKIP+SVM \\
	\midrule
	10 & 0.32304 & 0.23868 & 0.60991 & 0.61853 \\
	\midrule
	20 & 0.34567 & 0.23868 & 0.66810 & 0.68103 \\
	\midrule
	30 & 0.36419 & 0.24074 & 0.67456 & 0.68534 \\
	\midrule
	40 & 0.37448 & 0.23868 & 0.67672 & 0.69612 \\
	\midrule
	50 & 0.37654 & 0.23868 & 0.68103 & 0.69612 \\
	\midrule
	60 & 0.37654 & 0.23868 & 0.67672 & 0.69827 \\
	\midrule
	70 & 0.37860 & 0.23868 & 0.67672 & 0.69827 \\
	\midrule
	80 & 0.37860 & 0.23868 & 0.67887 & 0.69827 \\
	\midrule
	90 & 0.37860 & 0.23868 & 0.67887 & 0.69827 \\
	\midrule
	00 & 0.38271 & 0.23868 & 0.67887 & 0.69827 \\
	\bottomrule
	\end{tabular}
	\label{tab6}
	\end{table}

最後將表\ref{tab5}和表\ref{tab6}的結果,繪製在圖\ref{Fig5}上,在使用希臘的資料下,能明顯的發現SVM模型不管配上哪種詞向量模型,Recall的表現都比隨機森林這個模型好,對於不平衡的文本情緒分類,SVM處理優於隨機森林,而調整類別權重也會影響到Recall。\\
	
	\begin{figure}[H]
	\centering
	\includegraphics[scale=0.45]{output/希臘模型比較.png}
	\caption{希臘旅館模型(權重 vs. Recall)}
	\label{Fig5}
	\end{figure}
	
\newpage

\subsection{模型比較}

	將冰島的資料,對於不同模型組合下的結果,總結在表\ref{tab7},這裡多考慮兩種模型指標F1 Score和AUC Score,也能從表\ref{tab7}發現,Word2Vec的兩種方法再配上隨機森林或是GBDT時,模型的Recall和F1 Score表現沒有那麼好,甚至在與GBDT的組合上,表現還略低於詞袋模型和TF-IDF模型,但是當配上SVM的情況下,模型的表現有大幅度的提升,不管是Recall或F1 Score都是與其他兩種詞向量模型有著很大的差距。而在AUC Score上也是與SVM的組合為較佳,其次是與GBDT的組合,總而言之在冰島這筆資料得出Skip gram模型與SVM是最佳組合,在三個模型指標上都是最佳的表現。
	
	\begin{table}[H]
	\centering
	\caption{模型組合績效比較(冰島)}
	\begin{tabular}{|c|ccc|}
	\toprule
	模型組合 & Recall & F1 Score & AUC Score \\
	\midrule
	BOW+RF & 0.03941 & 0.05850 & 0.84961 \\
	\midrule
	BOW+SVM & 0.29045 & 0.28475 & 0.87043  \\
	\midrule
	BOW+GBDT & 0.18672 & 0.27314 & 0.89567  \\
	\midrule
	TFIDF+RF & 0.04356 & 0.06642 & 0.87279 \\
	\midrule
	TFIDF+SVM & 0.19087 & 0.27786 & 0.90735  \\
	\midrule
	TFIDF+GBDT & 0.14522 & 0.21472 & 0.89522  \\
	\midrule
	W2v(CBOW)+RF & 0.06564 & 0.11985 & 0.82272 \\
	\midrule
	W2v(CBOW)+SVM & 0.67833 & 0.42608 & 0.93825 \\
	\midrule
	W2v(CBOW)+GBDT & 0.13785 & 0.22743 & 0.92523  \\
	\midrule
	W2v(SKIP)+RF & 0.04157 & 0.07317 & 0.78807 \\
	\midrule
	W2v(SKIP)+SVM & 0.73960 & 0.43636 & 0.93996 \\
	\midrule
	W2v(SKIP)+GBDT & 0.13347 & 0.21981 & 0.93736  \\
	\bottomrule
	\end{tabular}
	\label{tab7}
	\end{table}
	
	將\ref{tab7}的表格資料,依照Recall的大小做排序,再經由長條圖\ref{Fig6}的呈現,也能清楚的看到模型組合比較結果。
	
	\begin{figure}[H]
	\centering
	\includegraphics[scale=0.65]{output/冰島模型比較圖.png}
	\caption{模型組合比較圖(冰島)}
	\label{Fig6}
	\end{figure}
	
	
	將試另一筆希臘的資料,對於不同模型組合下的結果,總結在表\ref{tab8}上,能發現其實模型組合的結果,與冰島這筆資料所呈現的大致相同,同樣的也是Skip gram模型與SVM為最佳組合,但唯一值得一提的是在Recall上的表現沒有像冰島的資料一樣,能夠破7成,而CBOW模型與SVM的組合也在F1 Score上,略表現好於SKIP模型與SVM的組合,將\ref{tab8}的表格資料,如同圖\ref{Fig6}的呈現方式,畫出圖\ref{Fig7}。
	
	\begin{table}[H]
	\centering
	\caption{模型組合績效比較(希臘)}
	\begin{tabular}{|c|ccc|}
	\toprule
	模型組合 & Recall & F1 Score & AUC Score \\
	\midrule
	BOW+RF & 0.06584 & 0.10000 & 0.85652  \\
	\midrule
	BOW+SVM & 0.38271 & 0.35722 & 0.85794 \\
	\midrule
	BOW+GBDT & 0.19135 & 0.28353 & 0.90168 \\
	\midrule
	TFIDF+RF & 0.07407 & 0.11538 & 0.87405  \\
	\midrule
	TFIDF+SVM & 0.24074 & 0.32584 & 0.90926  \\
	\midrule
	TFIDF+GBDT & 0.12962 & 0.19504 & 0.89760 \\
	\midrule
	W2v(CBOW)+RF & 0.04094 & 0.06477 & 0.84851 \\
	\midrule
	W2v(CBOW)+SVM & 0.67887 & 0.40905 & 0.92891 \\
	\midrule
	W2v(CBOW)+GBDT & 0.12567 & 0.20386 & 0.92700 \\
	\midrule
	W2v(SKIP)+RF & 0.02370 & 0.04573 & 0.80537 \\
	\midrule
	W2v(SKIP)+SVM & 0.69827 & 0.40769 & 0.93064  \\
	\midrule
	W2v(SKIP)+GBDT & 0.13146 & 0.21746 & 0.93009  \\
	\bottomrule
	\end{tabular}
	\label{tab8}
	\end{table}
	
	\begin{figure}[H]
	\centering
	\includegraphics[scale=0.65]{output/希臘模型比較圖.png}
	\caption{模型組合比較圖(希臘)}
	\label{Fig7}
	\end{figure}
	
	最後我們將畫出希臘這筆資料的ROC曲線(Receiver Operating Characteristic Curve),則挑選每種詞向量模型配上機器學習分類模型,其最好的AUC Score,當作挑選準則,如圖\ref{Fig8}。\\
	
	\begin{figure}[H]
	\centering
	\includegraphics[scale=0.45]{output/希臘roc.png}
	\caption{模型組合的ROC曲線(希臘)}
	\label{Fig8}
	\end{figure}

\newpage

\section{推薦旅館}

	對於那些留下負面評論的旅客,我們透過第二小節的結果,我們選擇使用Skip gram模型和SVM的組合,去預測出這類的旅客,在經由TF-IDF和Conditional Entropy的方法,我們找出每個旅館的關鍵字,在透過前面已經訓練好的Word2Vec的詞向量模型,去取得這些關鍵字的詞向量,以及負面旅客的評論也能利用此方式算出其詞向量,可由餘弦相似度去計算負面評論與特定旅館關鍵字的相似程度。這裡我們以圖\ref{Fig9}這個評論為例,這位旅客居住在冰島的Hotel Kjarnalundur(賈爾納倫杜酒店),得到很差的住宿體驗,透過訓練好的詞向量模型,將這則旅客評論轉為能夠運算的詞向量。\\
	
	\begin{figure}[H]
	\centering
	\includegraphics[scale=0.65]{論文內放的截圖/旅館負面評論.png}
	\caption{冰島旅館負面評論}
	\label{Fig9}
	\end{figure}

\newpage
	
\noindent 舉例用7個英文句子,來算其相似程度,當餘弦值越接近1的時候,代表與其評論越相似,反之越接近-1時越不相似,如表\ref{tab9},可以得知圖\ref{Fig9}有提到衛生環境的問題,還有像是旅館接待員粗魯的對待,所以第一句跟第四句,根據相似度計算出來的結果都是正的,而第二句和第五句稱讚旅館的正面回應與其負面評論結果大不相同,所以餘弦值算出來則為負的,最後是第七句提到的早餐不好,是個負面評論,但是與範例旅客留下的負面內容無關,所以算出來的餘弦值也是負的。
	
	\begin{table}[H]
	\centering
	\caption{與負面評論相似度計算示例}
	\begin{tabular}{|c|c|c|}
	\toprule
	& 範例英文句子 & cosine similarity \\
	\midrule
	1 & The room is dirty. & 0.23143  \\
	\midrule
	2 & The hotel is very good and highly recommended. & -0.15019 \\
	\midrule
	3 & Bad wifi signal. & 0.06375  \\
	\midrule
	4 & There is a rude receptionist. & 0.28184 \\
	\midrule
	5 & The location of this hotel is good. & -0.02177\\
	\midrule
	6 & The weather today is good. & -0.07548 \\
	\midrule
	7 & Breakfast is really bad. & -0.07792 \\
	\bottomrule
	\end{tabular}
	\label{tab9}
	\end{table}

	所以透過以上的方式,我們會由TF-IDF和Conditional Entropy得到84家冰島旅館個字代表的負面評論的30個關鍵字,同樣的每個關鍵字都會有一個詞向量,將這些關鍵字與旅客留下的負面評論,去計算餘弦值。這邊的做法是算出後每個旅館將會有30個餘弦值,在從其中挑出最大的5個做平均,則每個旅館都會算出一個負面相似得分,最後再向旅客推薦負面相似得分的較低的前10家旅館。而這裡取最大的5個餘弦值,原因是為了要避免旅客再遇到相似問題的旅館,甚至旅客可能抱怨旅館不只一個問題,所以取了較多的關鍵字。從表\ref{tab10}可以經由這樣方法得到推薦的旅館名稱,而優先推薦給這位旅客的是Hótel Kría(科里亞酒店)。\\
	
	\begin{table}[H]
	\centering
	\caption{相似得分較低的前10家冰島旅館}
	\begin{tabular}{|c|c|}
	\toprule
	冰島旅館名稱 & 負面相似得分 \\
	\midrule
	Hótel Kría(科里亞酒店) & 0.05993  \\
	\midrule
	Eyja Guldsmeden Hotel(艾加古斯米登酒店) & 0.06719 \\
	\midrule
	Midgardur by Center Hotels & 0.09090  \\
	\midrule
	Reykjavik Residence Apartment Hotel(雷克雅未克公寓酒店) & 0.09829 \\
	\midrule
	Icelandair Hotel Myvatn(米湖冰島之空酒店) & 0.09986\\
	\midrule
	Hali Country Hotel(哈利鄉村酒店) &  0.10310 \\
	\midrule
	Fosshotel Glacier Lagoon(冰河潟湖福斯酒店) & 0.10336 \\
	\midrule
	Eric the Red Guesthouse(埃里克紅旅館) & 0.10487 \\
	\midrule
	Guesthouse Carina(卡瑞納旅館) & 0.11001 \\
	\midrule
	Lilja Guesthouse(麗杰旅館) & 0.11315 \\
	\bottomrule
	\end{tabular}
	\label{tab10}
	\end{table}
	
\noindent 這裡也透過驗證的方式,去列出負面相似得分高分的前10家旅館,從表\ref{tab11}能看到這位旅客原先住的旅館Hotel Kjarnalundur(賈爾納倫杜酒店),位於不推薦的排名第四名,說明這樣方法的確能避開有類似問題的旅館。\\

	\begin{table}[H]
	\centering
	\caption{相似得分較高的前10家旅館}
	\begin{tabular}{|c|c|}
	\toprule
	冰島旅館名稱 & 負面相似得分 \\
	\midrule
	Puffin Hotel Vík(維克普費因酒店) & 0.24648  \\
	\midrule
	Loft - HI Hostel(閣樓- HI旅舍) & 0.23192 \\
	\midrule
	Guesthouse Bitra B \& B(比特拉住宿加早餐旅館) & 0.22482 \\
	\midrule
	Hotel Kjarnalundur(賈爾納倫杜酒店) &  0.21360 \\
	\midrule
	Stay Apartments Einholt(埃因霍特公寓酒店) &  0.21220 \\
	\midrule
	Sólheimahjáleiga Guesthouse(索爾海馬夏列卡農家樂) &  0.20980 \\
	\midrule
	Hotel Ísland – Spa \& Wellness Hotel & 0.20893 \\
	\midrule
	Hotel Hafnarfjall(哈芬納爾福爾酒店) &0.20876 \\
	\midrule
	Center Hotels Arnarhvoll & 0.20825 \\
	\midrule
	Guesthouse Fludir - Grund(弗洛蒂爾大地旅館) & 0.19636 \\
	\bottomrule
	\end{tabular}
	\label{tab11}
	\end{table}
	
\newpage

	同樣的用希臘的負面評論圖\ref{Fig10}為例,也能替這位在Metropolis Hotel(大都市酒店)有著不好旅遊體驗的旅客,為他在選定的70個希臘旅館中,推薦較適合的旅館,如表\ref{tab12}列出的前10家旅館。\\
	
	\begin{figure}[H]
	\centering
	\includegraphics[scale=0.65]{論文內放的截圖/旅館負面評論2.png}
	\caption{希臘旅館負面評論}
	\label{Fig10}
	\end{figure}
	
		\begin{table}[H]
	\centering
	\caption{相似得分較低的前10家希臘旅館}
	\begin{tabular}{|c|c|}
	\toprule
	希臘旅館名稱 & 負面相似得分 \\
	\midrule
	Athens Hawks(科里亞酒店) & 0.27512  \\
	\midrule
	Athinaiko Hotel(阿西恩尼克酒店) & 0.28312 \\
	\midrule
	The Athens Gate Hotel(雅典門酒店) & 0.28420 \\
	\midrule
	Minoa Athens Hotel(雅典彌諾阿酒店) & 0.29753 \\
	\midrule
	Centrotel Hotel(薩特洛泰酒店) & 0.29960 \\
	\midrule
	Babis Hotel(巴比酒店) &  0.30310 \\
	\midrule
	Evripides Hotel(艾弗瑞派德酒店) & 0.31334 \\
	\midrule
	Hotel Fresh(弗萊士酒店) & 0.32233 \\
	\midrule
	Adam's Hotel(亞當酒店) & 0.32247 \\
	\midrule
	Infinity City Boutique Hotel(無限城精品酒店) & 0.32740 \\
	\bottomrule
	\end{tabular}
	\label{tab12}
	\end{table}

%\end{document}










