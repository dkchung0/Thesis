%\input{preamble}
%%\usepackage{wallpaper}                                          % 使用浮水印
%%\CenterWallPaper{0.6}{images/ntpu.eps}                           % 浮水印圖檔
%\begin{document}
%\fontsize{12}{22pt}\selectfont
%\cleardoublepage
%%\thispagestyle{empty}
%\setlength{\parindent}{2em}
\chapter{文獻回顧}

\section{旅遊評論}

	Gretzel與Kyung (2008)說明了消費者所留下的旅遊評論,越來越重要,作者經由對於TripAdvisor的用戶進行調查,結果表明這些旅遊評論,對消費者主要用途是為其住宿決定提供有用的參考訊息,而另外也提到,這些旅遊評論對於旅館業者,提供了對於旅館營運和行銷有很大的啟示。
	
	Song等(2019)也做過實驗去驗證,使用TripAdvisor這個訂房網站,對於來自日本北海道7個不同區域的評論樣本提取滿意程度,與政府自己實施的旅客滿意度調查,通過Pearson相關係數,算出與其中6個區域的調查紀錄存在強烈的正相關,也說明了這些線上評論的可行性。
	
	透過Schuckert等(2015)統計2004到2013關於旅館評論的學術論文,並經由論文回顧以及分析,明確解釋對於旅館或研究人員如何利用旅館評論,去挖掘有用資訊,並進行電子商務策略。並經由訂房網站上香港旅館資訊,從資料能看到熱門旅館中超過7成都採用了線上回復管理,目的是阻止負面口碑的傳播,也能盡快去改進旅館內的缺點。
	
\newpage

\section{特徵提取}

	對於找出關鍵字,Ramos (2003)透過TF-IDF去尋找語料庫重要的詞,當作在查詢的關鍵字,而且證明了這個簡單的算法,很有效地能找到與這個關鍵字相關信息的文檔,對於查詢檢索系統有很大的幫助,但作者也指出TF-IDF有侷限性,TF-IDF不會考慮詞與詞之間的關係,每個單詞都是獨立的,所以無法處理同義字和單複數的問題。
	
	除了TF-IDF可以提取文本特徵,Chen與Wang (2012)提出一種基於聯合條件熵和遺傳算法的特徵提取方法,聯合條件熵是給定條件下的一組變量的不確定性度量,作者指出當這組特徵詞變量的不確定越小時,越能代表該文本的特徵詞,用此方法可以找到理想的文本特徵。
	
	但對於不同的文本,採用的方法也可能不太一樣,像是Zhao等(2017)也提出了一種基於Word2Vec和Textrank的關鍵字提取方法,則為了去解決社群網路短文本的問題,如TF-IDF這類的方法在社群網路上的表現不好,因為需要大量的長文本去計算詞頻和逆文檔頻率,作者適用的這套方法在Twitter的資料,表現較其他傳統提取關鍵字的方法還要好。
	
	Wang與Zhang (2017)提出一種完全基於深度學習找尋關鍵字的方法,運用雙向長短記憶循環神經網路,去自動提取關鍵字,其中作者是將不同n-gram的字詞,以頻率高低進行排序,將排名較高的當作關鍵字類的標籤,接著把關鍵字提取當作一個分類問題,去計算每個類標籤在文本的頻率,最後將高頻類的標籤當作最後的關鍵字,對於六種產品的中文評論進行的實驗表明,此方法有很高的關鍵字提取準確率。
	
	
\newpage
	
\section{模型表現}

	Mikolov等(2013)提出了Word2Vec,以及二種不同的訓練方式來計算詞向量,分別是CBOW以及Skip-gram,比起傳統方法,Word2Vec能夠從大量的資料中學習到詞向量,且訓練速度也得到很大的提升,最後得到詞向量還能夠去表達在句中的詞義,模型甚至能做到詞向量的線性運算。Mikolov等(2013)是再對於前面的Skip-gram的模型去改進,為了去解決當詞庫太大時,出現計算複雜度偏高的問題,作者提出了Hierarchical Softmax和負採樣的方式去提高模型的運算效率。
	
	Mitra等(2007)提出了一個LS-SVM的模型,作者有說明分類器是使用RBF Kernel函數,透過潛在語意分析的技術去獲得語義的信息 ,再去進行文本分類,研究結果則是標題分類準確率高達99.9%
	
	Patel與Meehan (2021)提到為了想要去分析虛假Reddit上的假新聞,運用到文字模型和機器學習分類模型去做預測,作者的文字模型選用了CountVectorizer和TF-IDF,機器學習模型則使用邏輯斯迴歸、MultinominalNB、SVM,而作者的研究結果是指出運用CountVectorizer配上MultinominalNB會得到最高的準確率(Accuracy)。
	
	Huang等(2021)說明了評論文本的價值性,作者為了解決當兩種情感極性同時出現,情感分類邊界難以去區分是哪一種類的問題,透過潛在狄利克雷分配的主題模型,進行文本特徵數量的拓展,再透過SVM、RF、GDBT、XGBoost去進行文本分類,實驗結果則是表明這些步驟有助於去增強情感邊界的分類能力,模型則是透過TF-IDF提取特徵後由GBDT分類表現最佳。
	
\newpage
	
\section{推薦方法}

	Xia等(2015)說明了在分類或分群上,餘弦相似度是一種優先會考慮的方法,原因是不只理論很直觀,表現出來的結果又有效,當相似性度量的定義明確時,能夠處理很多種類的問題。作者也提出了一種基於餘弦相似度改良的方法,稱作餘弦相似度集成(CSE),透過多個弦相似度學習器去計算相似度,也經過作者的數據集,去證明CSE優於其他計算相似度的方法。
	
	Schafer等(2007)在論文中介紹了協同過濾的核心概念、自適應網絡的主要的用途,也提及協同過濾的理論部分,以及怎麼使用這個方法,去設計一個推薦系統。也說明了協同過濾這個方法,是如何讓自適應網絡的個性化技術變得強大,協同過濾是使用其他人的意見過濾或是評估項目的過程,當擁有大量數據時,推薦系統能夠更精準去推薦。
	
	
	
%\end{document}




