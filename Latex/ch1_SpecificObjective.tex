%\input{preamble}
%%\usepackage{wallpaper}                                          % 使用浮水印
%%\CenterWallPaper{0.6}{images/ntpu.eps}                           % 浮水印圖檔
%\begin{document}
%\fontsize{12}{22pt}\selectfont
%\cleardoublepage
%\setlength{\parindent}{2em}
\chapter{緒論}

\section{研究背景和動機}
		
	近年來網路蓬勃發展,人們旅遊訂飯店的模式,從以前打電話預訂的模式變成網路上能夠即時訂房,在這生活步調很快的年代,使用者能應需求快速得到詳細的資料,成為各家訂房業者的目標。隨著線上訂房的需求的增加,訂房網站越來越多,每家訂房網站主打的服務也大不相同,例如:入住時再付款的Booking.com,主打民宿優惠的Airbnb,還有近年快速成長的Agoda,網站因應不同的消費者需求和主打的使用者去提供客製化優惠。
	
	現今許多產業都開始運用機器學習的技術,去解決生活周遭所遇到的問題,Gretzel與Kyung (2021)提到因為COVID-19的疫情持續影響著人們,旅館業者開始去依賴這樣的技術去實行解決方案,Koo等(2021)提到越來越多旅館將人工智能和機器人技術融入旅館的經營中,對於旅館的生態產生很大的影響和改變,成為一種新的商業模式。所以各家旅館想透過這些資訊技術,比競爭對手更早一步滿足現代消費者的期望。其中訂房網站的旅客評分和評論系統,成為比較各間旅館好壞的一個標準,Stringam等(2010)也驗證了旅行者評論的重要性,旅客的回饋成為旅館業者所注重的部分。為了在網站呈現給使用者過去高分的體驗以及較多正向評論,旅館業者會聚焦在給出負面評論的旅客,對於其負面評論提到的問題去做改善,不同國家旅館業者遇到的問題也大不相同,對於應改善的問題也不一樣。
	
\newpage
	
	Eberendu (2016)提到隨著創新的技術出現,像是社群網路伴隨著大量複雜的非結構型資料,其資料不再像是傳統資料那樣具有標準的格式和結構,例如:文字、圖像、聲音、影片、XML,這些都無疑增加了資料分析和模型建置的難度。非結構型資料的出現,也使得自然語言處理、影像辨識、語音辨識等領域,漸漸被人們開始廣泛的研究,而除了改善既有的機器學習技術,隨著電腦運算效能提高的同時,深度學習的技術也重新浮上檯面。
	
	Mostafa (2020)蒐集多個訂房網站的資料對埃及在線旅遊評論進行分析。其使用詞向量模型TF-IDF去取得特徵,採用支持向量機、單純貝氏分類器、決策樹,做為分類模型預測出負面情感,結果為準確率下單純貝氏分類器表現的最好。但作者未考慮到詞義可能會影響預測結果,以及負面評論佔的比例較低時,用準確率來看模型表現會失準。
	
	Fazzolari與Petrocchi (2018)提到運用關聯規則學習算法去建置推薦系統。收集義大利的評論,透過挖掘評論者的國籍和最常訪問的地區之間的關聯性,為不同的評論者推薦首選的地區。但是去哪個地區旅遊其實是旅遊者主觀的意識,對於那種已經決定要去哪個地區旅遊的人較不適用,這些人會比較需要的是,推薦該地區一間合適的旅館。
	
\newpage
	 
\section{研究目的}
	
	本研究著重旅客在訂房網站上給出的低分以及負面評論,針對特定國家所有的旅館資料,尤其是文字型資料進行分析,欲以線上訂房網站特定國家旅館使用者評論。第一部分運用不同的資料探勘技術挖掘導致負面評論的關鍵字,使得旅館業者能制定改善計畫,去提升該旅館在訂房網站的分數表現,減少負面評論貼文數。第二部分會使用不同的詞向量模型,配合機器學習的分類模型,去預測同時給出低分和較於負面的評論的旅客,目的為訓練出特定國家的情感模型,透過選擇最佳的模型,在大量的評論資料中抓出真實負面評論的旅客,能夠優先處理這些旅客的需求,對旅店的缺點做出立即的改善。第三部分則是運用餘弦相似度計算,提出一個推薦旅館的方法,對於負面評論的旅客,當下次旅遊該國家時,推薦幾間較為合適的旅館,避免上一次遇到的問題再次發生。
	
%\end{document}

